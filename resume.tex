%!TEX TS-program = xelatex
%!TEX encoding = UTF-8 Unicode
% Awesome CV LaTeX Template for CV/Resume
%
% This template has been downloaded from:
% https://github.com/posquit0/Awesome-CV
%
% Original author:
% Claud D. Park <posquit0.bj@gmail.com>
% http://www.posquit0.com
%
% Modifications by:
% Pranav Srinivasan <pranavsankars@g.ucla.edu>
%
% Template license:
% CC BY-SA 4.0 (https://creativecommons.org/licenses/by-sa/4.0/)
%


%-------------------------------------------------------------------------------
% CONFIGURATIONS
%-------------------------------------------------------------------------------
% A4 paper size by default, use 'letterpaper' for US letter
\documentclass[12pt, letterpaper]{awesome-cv}
\usepackage{graphicx}

% Configure page margins with geometry
\geometry{left=0.5in, top=0.5in, right=0.5in, bottom=0.5in}

% Specify the location of the included fonts
\fontdir[fonts/]

% Color for highlights
% Awesome Colors: awesome-emerald, awesome-skyblue, awesome-red, awesome-pink, awesome-orange
%                 awesome-nephritis, awesome-concrete, awesome-darknight
\colorlet{awesome}{awesome-skyblue}
%\colorlet{awesome}{awesome-darknight}
% Uncomment if you would like to specify your own color
% \definecolor{awesome}{HTML}{CA63A8}

% Colors for text
% Uncomment if you would like to specify your own color
% \definecolor{darktext}{HTML}{414141}
% \definecolor{text}{HTML}{333333}
% \definecolor{graytext}{HTML}{5D5D5D}
% \definecolor{lighttext}{HTML}{999999}

% Set false if you don't want to highlight section with awesome color
\setbool{acvSectionColorHighlight}{true}

% If you would like to change the social information separator from a pipe (|) to something else
\renewcommand{\acvHeaderSocialSep}{\quad\textbar\quad}

\makeatletter
\patchcmd{\@sectioncolor}{\color}{\mdseries\color}{}{}
\makeatother

%-------------------------------------------------------------------------------
%	PERSONAL INFORMATION
%	Comment any of the lines below if they are not required
%-------------------------------------------------------------------------------
% Available options: circle|rectangle,edge/noedge,left/right
% \photo[rectangle,edge,right]{profile}
\name{}{Pranav Sankar Srinivasan}
% \position{Software Engineer}
% \address{address}

\mobile{(805) 624-4604}
\email{pranavsankars@g.ucla.edu}
\homepage{pranavs2001.github.io}
\github{pranavs2001}
\linkedin{pranavsankars}
% \gitlab{gitlab-id}
% \stackoverflow{SO-id}{SO-name}
% \twitter{@twit}
% \skype{skype-id}
% \reddit{reddit-id}
% \extrainfo{extra informations}

%-------------------------------------------------------------------------------
\begin{document}
% \vspace*{-0.35in}
% Print the header with above personal informations
% Give optional argument to change alignment(C: center, L: left, R: right)
\makecvheader
% \vspace*{-0.1in}
% Print the footer with 3 arguments(<left>, <center>, <right>)
% Leave any of these blank if they are not needed
% \makecvfooter
%   {\today}
%   {Henry Trinh~~~·~~~Résumé}
%   {\thepage}

%-------------------------------------------------------------------------------
%	CV/RESUME CONTENT
%	Each section is imported separately, open each file in turn to modify content
%-------------------------------------------------------------------------------

% EDUCATION
\cvsection{Education}
\begin{cventries}
  \cventry
    {B.S. Computer Engineering, GPA 3.7/4.0} % Degree
    {University of California, Los Angeles} % Institution
    {Los Angeles, CA} % Location
    {June 2023} % Date(s)
    {
      \begin{cvitems} % Description(s) bullet points
      \item {\textbf{Honors} --- \entrydatestyle{Fast Track Honors Program, Dean’s Honor List}}
        \item {\textbf{Relevant Coursework} --- \entrydatestyle{Data Structures \& Algorithms, Software Construction, Operating Systems, Computer Architecture, Internet of Things, Digital Logic Design}}
      \end{cvitems}
    }
    %{
    %  \vspace{-0.11in}
    %  \begin{cvskills}
    %  \item {\textbf{Relevant Coursework} --- \entrydatestyle{Data Structures and Algorithms, Software %Construction, Operating Systems, Computer Architecture, Internet of Things, \\Logic Design of Digital %Systems}}
    %    \cvskill
    %    {Relevant Courses} % Type
    %      {Data Structures and Algorithms, Software Construction, Operating Systems, Computer %Architecture, Internet of Things, \\Logic Design of Digital Systems} % Skillset
    %  \end{cvskills}
    %}
  \vspace{-0.2in}
\end{cventries}

% SKILLS
\cvsection{Skills}
\begin{cvskills}
  \cvskill
    {Languages} % Type
    {JavaScript, C/C++, Python, Swift, HTML, CSS} % Skillset

  \cvskill
    {Frameworks + Tools} % Type
    {React, Firebase, Bootstrap, Git, Postman, Node Package Manager, UIKit, Cocoapods} % Skillset
  
%  \cvskill
%    {Technologies} % Type
%    {Full Stack, iOS, Databases, Machine Learning, Computer Vision, SLAM} % Skillset
    
\end{cvskills}

% EXPERIENCE
\cvsection{Experience}
\begin{cventries}
  \cventry
    {Software Engineer Intern} % Job title
    {Tesla} % Organization
    {Fremont, CA} % Location
    {Summer 2021} % Date(s)
    {
      % \begin{cvitems} % Description(s) of tasks/responsibilities
      %   \item {Interning in Fall 2021}
      % \end{cvitems}
    }
  \vspace{-0.07in}

  \cventry
    {Undergraduate Researcher - HTML, CSS, Python, Flask} % Job title
    {Robotics and Mechanisms Laboratory (RoMeLa)} % Organization
    {Los Angeles, CA} % Location
    {October 2019 - Present} % Date(s)
    {
      \begin{cvitems} % Description(s) of tasks/responsibilities
        \item {Developing website to interface with an autonomous cooking robot and using Flask to host the website locally on the central Raspberry Pi unit}
        \item {Implemented serial communication capability between Raspberry Pi and Arduino to monitor sensor data}
      \end{cvitems}
    }

\end{cventries}

%Leadership
\cvsection{Leadership}

\begin{cventries}
  \cventry
    {External Vice President} % Affiliation/role
    {UCLA IEEE} % Organization/group
    {Los Angeles, CA} % Location
    {April 2021 - Present} % Date(s)
    {
      \begin{cvitems} % Description(s) of experience/contributions/knowledge
        \item {Organizing the largest hardware hackathon on the west coast, IDEA Hacks (100+ participants)}
        \item {Previously Corporate Relations Officer: hosted industry events with prominent partners for 300 members and fundraised \$7000 for IDEA Hacks}
      \end{cvitems}
    }
\end{cventries}


% PROJECTS
\cvsection{Projects}
\begin{cventries}
  % \cventry
  %   {} % Empty position
  %   {Simultaneous Location and Mapping (SLAM) Tool} % Project
  %   {} % Empty location
  %   {} % Empty date
  %   {
  %     \vspace{-0.2in}
  %     \begin{cvitems} % Description(s) bullet points
  %       \item {Developed tool to generate a three-dimensional map of an environment by analyzing any given video}
	% 	    \item {Performed video and image processing using OpenCV to extract features from images and render their points in 3-D space}
  %     \end{cvitems}
  %   }

    \cventry
        {JavaScript, React, Firebase, Postman, Chrome API} % Empty position
        {Code Crumbs} % Project
        {} % Empty location
        {} % Empty date
        {
          % \vspace{-0.2in}
          \begin{cvitems} % Description(s) bullet points
          	\item {Developed Chrome extension that allows users to track search history in detail to aid the software development process}
    		    \item {Implemented sign-in/login functionality using Firebase Authentication and capability to toggle the extension on/off}
          \end{cvitems}
        }

  \cventry
    {JavaScript, React, Python, Firebase, mealDB API} % Empty position
    {\href{https://insta-chef-ba8dc.web.app}{Insta-Chef}} % Project
    {} % Empty location
    {} % Empty date
    {
      % \vspace{-0.2in}
      \begin{cvitems} % Description(s) bullet points
      	\item {Created web application that enables users to discover new culinary recipes by name or ingredient and track all items in kitchen}
      	\item {Constructed React components to display various information and added front-end features to simplify the process of tracking ingredients}
      \end{cvitems}
    }
    
    \cventry
    {JavaScript, Bootstrap, Spotify API} % Empty position
    {\href{https://discovr.netlify.app}{DisCovr}} % Project
    {} % Empty location
    {} % Empty date
    {
      % \vspace{-0.2in}
      \begin{cvitems} % Description(s) bullet points
      	\item {Built a web application that enables Spotify users to find new songs not based on their prior listening history}
		\item {Devised a song search query algorithm that randomly selects new songs efficiently and produced an intuitive interface for easy navigation of the app}
      \end{cvitems}
    }
    
    \cventry
    {C, Arduino - Runner-up for “Sustainability Award” amongst 60 participating teams} % Empty position
    {Power Shower} % Project
    {} % Empty location
    {} % Empty date
    {
      % \vspace{-0.2in}
      \begin{cvitems} % Description(s) bullet points
      	\item {Ideated and prototyped an IoT smart shower timer hub in 36-hour hackathon to raise awareness of water consumption in college dorms}
		\item {Integrated Bluetooth communication between Arduinos to allow for devices to join the IoT ecosystem}
		%\item {Runner-up for the “Sustainability Award” amongst 60 participating teams}
      \end{cvitems}
    }
\end{cventries}

%-------------------------------------------------------------------------------
\end{document}
